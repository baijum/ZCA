% latex include file for docutils latex writer
% --------------------------------------------
%
% CVS: $Id: style.tex,v 1.1 2004/01/23 21:17:32 faassen Exp $
%
% This is included at the end of the latex header in the generated file,
% to allow overwriting defaults, although this could get hairy.
% Generated files should process well standalone too, LaTeX might give a
% message about a missing file.

% donot indent first line of paragraph.
%\setlength{\parindent}{0pt}
%\setlength{\parskip}{5pt plus 2pt minus 1pt}

% sloppy
% ------
% Less strict (opposite to default fussy) space size between words. Therefore
% less hyphenation.
%\sloppy

% fonts
% -----
% times for pdf generation, gives smaller pdf files.
%
% But in standard postscript fonts: courier and times/helvetica do not fit.
% Maybe use pslatex.
%\usepackage{times}

% pagestyle
% ---------
% headings might put section titles in the page heading, but not if
% the table of contents is done by docutils.
% If pagestyle{headings} is used, \geometry{headheight=10pt,headsep=1pt}
% should be set too.
%\pagestyle{plain}
%
% or use fancyhdr (untested !)
%% \usepackage{fancyhdr}
%% \pagestyle{fancy}
%% \addtolength{\headheight}{\baselineskip}
%% \renewcommand{\sectionmark}[1]{\markboth{#1}{}}
%% \renewcommand{\subsectionmark}[1]{\markright{#1}}
%% \fancyhf{}
%% \fancyhead[LE,RO]{\bfseries\textsf{\thepage}}
%% \fancyhead[LO]{\textsf{\footnotesize\rightmark}}
%% \fancyhead[RE]{\textsc{\textsf{\footnotesize\leftmark}}}
%\fancyfoot[LE,RO]{\bfseries\textsf{\scriptsize Docutils}}
%\fancyfoot[RE,LO]{\textsf{\scriptsize\today}}

% geometry
% --------
% = papersizes and margins
%\geometry{a4paper,twoside,tmargin=1.5cm,
%          headheight=1cm,headsep=0.75cm}

% Do section number display
% -------------------------
%\makeatletter
%\def\@seccntformat#1{}
%\makeatother
% no numbers in toc
%\renewcommand{\numberline}[1]{}


% change maketitle
% ----------------
%\renewcommand{\maketitle}{
%  \begin{titlepage}
%    \begin{center}
%    \textsf{TITLE \@title} \\
%	Date: \today
%    \end{center}
%  \end{titlepage}
%}
% \usepackage[final]{graphicx}
\graphicspath{{./pics/}{./paropt/pics/}}
\usepackage[usenames]{xcolor}
\usepackage{enumitem}


% Main style definition
% ---------------------
% ISU standard handbook.
%\usepackage[times,fancybot,firamono]{subook} % looks less good than inconsolata
%\usepackage[times,inconsolata,handbook,irnitu]{subook} % Looks Good
\usepackage[irnitu,times,fancybot,inconsolata,smalltitles,microtyping]{subook}
\usepackage{shellesc}
\usepackage[final]{minted}
%\usemintedstyle{emacs}
\usemintedstyle{tango}
%\usemintedstyle{trac} % better (more bold faces}
%\usemintedstyle{manni} % pastel colors
%\usemintedstyle{bw}
\definecolor{mintedbg}{rgb}{0.95,0.95,0.95}
\setminted{breaklines=true,fontsize=\small,funcnamehighlighting=true,python3=false}

%\newminted{prolog}
\protect\newminted[ex]{text}{style=bw}

\usepackage{scrextend}
\changefontsizes{14pt}
%\usepackage[times]{subook}

%\setcitestyle{}

% Заменить в коде кой-что на математические формулы
% http://www.tug.org/texlive/Contents/live/texmf-dist/doc/latex/base/alltt.pdf

% Требования к оформлению Типографии ИГУ
% http://lawinstitut.ru/ru/about/services/izdatelstvo/trebovaniya.html
%

% ISU standard monograph (less restrictive and more artistic)
% \usepackage[monograph,mag,times,smalltitles,fancybot,listbib,ptfonts,microtyping]{subook}

% Some artistism for monograph
% ----------------------------
%\makeatletter{}
%\renewcommand\su@chapter@font{\sffamily\sfcpshape\bfseries}
%\renewcommand\su@chapter@font@size{\LARGE}
%\makeatother{}
%\floatname{algorithm}{Процедура}
%\renewcommand{\listalgorithmname}{Список процедур}
%\renewcommand\cftsecnumwidth{5ex}
%\tolerance=5000
%\usepackage[final]{hyperref}

\definecolor{mygreen}{rgb}{0,0.6,0}
\definecolor{mygray}{rgb}{0.5,0.5,0.5}
\definecolor{mymauve}{rgb}{0.58,0,0.82}


\usepackage{tikz}
\usetikzlibrary{arrows,arrows.meta,shapes}
\usetikzlibrary{shadows}
\newcommand*\keystroke[1]{%
  \tikz[baseline=(key.base)]
    \node[%
      draw,
      fill=white,
      drop shadow={shadow xshift=0.25ex,shadow yshift=-0.25ex,fill=black,opacity=0.75},
      rectangle,
      rounded corners=4pt,
      inner sep=1pt,
      line width=0.7pt,
      font=\footnotesize\sffamily
    ](key) {~#1~\strut}%
  ;%
}

\long\def\rem#1{}
%\def\AR{{\em Прим.~авторов~пособия}}
\def\emphbib#1{#1}
\newenvironment{questions}{\subsubsection*{Вопросы для самопроверки}\begin{enumerate}\itemsep0pt minus 0.3pt\parskip0pt plus 0.3pt}{\end{enumerate}}

\newtheorem{example}{Пример}[chapter]
\hypersetup{
    bookmarks=true,         % show bookmarks bar?
    unicode=true,           % non-Latin characters in Acrobat’s bookmarks
    pdftoolbar=true,        % show Acrobat’s toolbar?
    pdfmenubar=true,        % show Acrobat’s menu?
    pdffitwindow=false,     % window fit to page when opened
    pdfstartview={FitH},    % fits the width of the page to the window
    pdftitle={Компьютерные науки, часть 4},    % title
    pdfauthor={Евгений Александрович Черкашин},     % author
    pdfsubject={Методическое пособие},   % subject of the document
    pdfcreator={EMACS-24.4:AuCTeX},   % creator of the document
    pdfproducer={LuaLaTeX}, % producer of the document
    pdfkeywords={Искусственный интеллект} {Логическое
      программирование} {Планирование действий} {Удовлетворение
      ограничений} {Компьютерная алгебра} {Принцип максимума} {Оптимальное управление}, % list of keywords
    pdfnewwindow=true,      % links in new window
    colorlinks=true,       % false: boxed links; true: colored links
    linkcolor=[rgb]{0 0.4 0.1},          % color of internal links (black)
    citecolor=blue,        % color of links to bibliography
    filecolor=black,      % color of file links
    urlcolor=[rgb]{0.3 0.0 0.3}           % color of external links
}

%\renewcommand{\headrulewidth}{1pt}

\clubpenalty=3000
\widowpenalty=3000
%\brokenpenalty=10000
%\floatingpenalty=10000

%% \setdefaultlanguage{russian}
%% \setmainlanguage{russian}
%% \setotherlanguage{english}

%\newenvironment{mygroup}{}{}

\renewcommand\baselinestretch{1.5} % FIXME Какой интервал тут реально получится?
\newcommand{\eeng}[1]{\emph{\foreignlanguage{english}{#1}}}

\definecolor{rclr}{rgb}{0.5,0.1,0.1}
\definecolor{eclr}{rgb}{0,0.5,0.5}
\colorlet{acolor}{blue}
\colorlet{rcolor}{red}
\definecolor{ncolor}{rgb}{0.5,0.5,0.1}
\newcommand{\aaa}[2][acolor]{\noindent\textcolor{eclr}%
{+\ [}\textcolor{#1}{#2}\textcolor{eclr}{]}}
\newcommand{\rrr}[2][rcolor]{\noindent%
\textcolor{eclr}{-\ [}\textcolor{#1}{#2}\textcolor{eclr}{]}}
\newcommand{\nnn}[2][ncolor]{\noindent%
\textcolor{eclr}{!\ [}\textcolor{#1}{#2}\textcolor{eclr}{]}}
\newcommand{\goforth}[1]{$\,\hookrightarrow$\pageref{#1}}

% \begin{figure}
%   \centering
%   \def\svgwidth{\columnwidth}
%   \includesvg{image}
% \end{figure}

\parskip=0pt plus 0.3pt
\renewcommand{\refname}{Рекомендуемая литература} % ... also
\renewcommand{\bibname}{\refname}

%%% Local Variables:
%%% mode: latex
%%% TeX-master: "zca-hb-ru-proto"
%%% TeX-engine: luatex
%%% End:
